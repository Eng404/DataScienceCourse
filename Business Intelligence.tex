

\documentclass[12pt]{article}

%opening
\title{Data Science}
\author{Kevin O'Brien}

\begin{document}
\section*{Business Intelligence}
	
\begin{itemize}
\item In many service industries today, collecting and managing information about the customer is an important method of increasing revenue for a company. A customer database is a computer file system that tracks the customers of a business. This database can be configured to track specific customer profiles. This collection of information can be used to manage loyalty programs, warranty programs, customer issues, and retention programs. Each database is designed to provide a company with marketing information that can be used to retain and expand the customer base.

 

\item A customer loyalty program is designed to manage the spending patterns of an individual. These loyalty programs provide benefits back to the customer including travel, coupons, or cash. This type of program relies on a customer database that tracks the spending habits of the customer. Typically these spending habits are transformed into awards points, which can accumulate into actual cash benefits.

 

\item A warranty program is customer insurance policy on services and products. The policy gives customers a way to return products in the event of defects or failures that may occur over time. The warranty program is typically used for expensive products such as appliances, automobiles, and homes. This type of program depends on a customer database that tracks the individuals who have paid for this type of service.

 
\item Typically, a customer is required to pay an additional fee for an extended warranty when he buys a product. The customer database tracks the value and longevity of the insurance policy. Each warranty product is purchased for a specific duration on an individual product. Most warranty databases include vast amounts of data on products, defects, and customers.

\item A customer issues database is a special issues-tracking set of information that is typically used by large companies. This type of database is generally used by companies that rely particularly on customer service to grow their business. This includes banks, cable companies, and internet providers.

 
\item 
An issues database tracks customers based on support calls for the products that they use. The customer database for support centers provides a detailed review of the issues that have been claimed by a specific customer and the issue’s resolution. This provides accountability for the company to help with future customer relationships.

 

\item Many banks use a customer database to track the products used by each customer. This includes mortgages, car loans, and credit cards. The customer retention database is used by the banking industry to help sell additional banking products. Typically when an individual calls the bank, a customer profile is presented to the attendant that outlines the products of the customer. This helps the attendant sell additional products based on the customer’s needs.
\end{itemize}
\end{document}